\documentclass[../report.tex]{subfiles}
\begin{document}

The project Github repository is publicly available at

\section{Source Code}
Relevant parts of the source code is shown in these appendices to enable the reader to better understand the functionality of the modules.

\subsection{PWM control Module}
\begin{lstlisting}[language=VHDL]

architecture Behavioral of pwm_control is
signal count_sig: unsigned(30 downto 0) := (others => '0');
signal dutyCycleVal_sig : unsigned(12 downto 0) := "0101110001000";
begin

process(clk_inc)
begin

    if ( rising_edge(clk_inc) ) then
        count_sig <= count_sig + 1;
    end if;

    -- if 10 [ms] have passed evaluate
    if ( count_sig >= 100000 ) then
        if ( pwm_inc = "01" ) then
            dutyCycleVal_sig <= dutyCycleVal_sig + 1; -- DECREASE PWM
        elsif ( pwm_inc = "10" ) then -- and dutyCycleVal_sig > 0
            dutyCycleVal_sig <= dutyCycleVal_sig - 1; -- INCREASE PWM
        end if;
        count_sig <= (others => '0');
    end if;

end process;

dutyCycleVal <= std_logic_vector(dutyCycleVal_sig);


end Behavioral;

\end{lstlisting}

\subsection{PS Info Bits Module}
\begin{lstlisting}[language=VHDL]

architecture Behavioral of PS_info_3_bits is

begin
pwm_inc     <= data(1 downto 0);
battery_lvl <= data(2);

end Behavioral;

\end{lstlisting}

\subsection{Control Loop}
The code for implementation of BRAM communication in C is inspired by the code supplied during lecture 6 held by Beck Steckhahn-Strohmer.
\begin{lstlisting}[language=C, showstringspaces=false]

/* Include Files */
#include <stdio.h>
#include "platform.h"
#include "xil_printf.h"
#include "xil_types.h"
#include "xparameters.h"
#include "bram.h"
#include "xadcps.h"
#include "xstatus.h"
#include <unistd.h>
#include "core_portme.h"
/* Constant Definitions */
#define XADC_DEVICE_ID 		XPAR_XADCPS_0_DEVICE_ID
/* Function Prototypes */
static int XAdcGetVoltageFromRegisters(u16 XAdcDeviceId, int *voltage, int register_offset);
static int XAdcFractionToInt(float FloatNum);
static int float2Int(float f);
/* Macros (Inline Functions) Definitions */
#define printf xil_printf /* Small foot-print printf function */
// converting RAW data from external sourse to voltage
#define XAdcPs_ExternalRawToVoltage(AdcData)\
	((((float)(AdcData))* (1.0f))/65536.0f)
#define V_BAT_EQ 3.9375
#define I_CHARGE_EQ 15.75
#define V_BAT_0 1.1
/* Variable Definitions */
static XAdcPs XAdcInst;      /* XADC driver instance */
/**************************/
int main(void)
{
	init_platform();
	initMemory(); // BRAM
	//XADC
	int voltage_reading_A0, voltage_reading_A1 = 0; // Result is in mV going from 0V = 0 to 1000 = 3.3V
	int pwm = 0b00;
	print("Hello XADC!\n");
	int Status;
	// Get user input
	
	float v_nominal;
	xil_printf("Enter the nominal battery voltage [V]:\r\n");
	scanf("%f", &v_nominal);
	xil_printf("V_nominal = %d.%d [V]\r\n",(int)v_nominal, XAdcFractionToInt(v_nominal));

	int capacity;
	xil_printf("Enter the battery capacity [mAh]:\r\n");
	scanf("%d", &capacity);
	xil_printf("Capacity = %d [mAh]:\r\n", capacity);

	float k;
	xil_printf("Enter the charge speed( k = [0.1, 0.5] ):\r\n");
	scanf("%f", &k);
	xil_printf("k = %d.%d \r\n",(int)k, XAdcFractionToInt(k));

	float c;
	xil_printf("Enter the rate of charge (C):\r\n");
	scanf("%f", &c);
	xil_printf("Rate of charge = %d.%d\r\n",(int)c, XAdcFractionToInt(c));

	Status = XAdcGetVoltageFromRegisters(XADC_DEVICE_ID, &voltage_reading_A0, 1); // read from A0
	XAdcGetVoltageFromRegisters(XADC_DEVICE_ID, &voltage_reading_A1, 9); // read from A1
	if (Status != XST_SUCCESS) {
		printf("Failed reading from ADC");
		return XST_FAILURE;
	}

	// Converts to V
	float v_adc_1 = (voltage_reading_A0*3.3)/1000.0;
	float v_adc_2 = (voltage_reading_A1*3.3)/1000.0;

	float i_charge = I_CHARGE_EQ * (v_adc_1 - v_adc_2) * 1000;
	float v_bat = V_BAT_EQ * v_adc_2;
	float i_target = capacity * c * k;
	float t_left = capacity / i_charge;

	xil_printf("Target current = %d.%d [mA]", (int)i_target, XAdcFractionToInt(i_target));

	int led, v_bat_lvl;

	while (v_bat < v_nominal)
	{
		XAdcGetVoltageFromRegisters(XADC_DEVICE_ID, &voltage_reading_A0, 1); // read from A0
		XAdcGetVoltageFromRegisters(XADC_DEVICE_ID, &voltage_reading_A1, 9); // read from A1

		v_adc_1 = (voltage_reading_A0 * 3.3) / 1000.0;
		v_adc_2 = (voltage_reading_A1 * 3.3) / 1000.0;

		i_charge = I_CHARGE_EQ * (v_adc_1 - v_adc_2) * 1000.0;
		t_left = capacity / i_charge;

		pwm = 0b00;
		if (i_charge > i_target)
		{
			pwm = 0b01; //decrease
		}

		if (i_charge < i_target)
		{
			pwm = 0b10; //increase
		}

		v_bat = V_BAT_EQ * v_adc_2;

		v_bat_lvl = (v_bat - V_BAT_0) *100 / (v_nominal - V_BAT_0);
		if (v_bat_lvl > 20)
			led = 0b1;
		else
			led = 0b0;

		MYMEM_u(0)=CombineForBram(led, pwm); // using memory addr 0

		xil_printf("Vbat: %d.%d, Time_left: %d.%d, itarget: %d.%d, icharge: %d.%d\n\r", (int)v_bat, float2Int(v_bat),(int)t_left, float2Int(t_left), (int)i_target, float2Int(i_target), (int)i_charge, float2Int(i_charge));
	}
	xil_printf("Done charging");
	cleanup_platform();
	return XST_SUCCESS;
}

/*******************************************/
/*	Reading voltage */
int XAdcGetVoltageFromRegisters(u16 XAdcDeviceId, int *voltage, int register_offset)
{
	int Status;
	XAdcPs_Config *ConfigPtr;
	u32 VccPdroRawData;
	float VccPintData;
	XAdcPs *XAdcInstPtr = &XAdcInst;

	/*
	 * Initialize the XAdc driver.
	 */
	ConfigPtr = XAdcPs_LookupConfig(XAdcDeviceId);
	if (ConfigPtr == NULL) {
		return XST_FAILURE;
	}
	XAdcPs_CfgInitialize(XAdcInstPtr, ConfigPtr, ConfigPtr->BaseAddress);

	/*
	 * Self Test the XADC/ADC device
	 */
	Status = XAdcPs_SelfTest(XAdcInstPtr);
	if (Status != XST_SUCCESS) {
		return XST_FAILURE;
	}

	/*
	 * Read the A1 input Voltage Data from the
	 * ADC data registers.
	 */
	VccPdroRawData = XAdcPs_GetAdcData(XAdcInstPtr, XADCPS_CH_AUX_MIN+register_offset);
	VccPintData = XAdcPs_ExternalRawToVoltage(VccPdroRawData);


	(*voltage) = XAdcFractionToInt(VccPintData);

	return XST_SUCCESS;
}

/*******************************************************/
/*
*
* This function converts the fraction part of the given floating point number
* (after the decimal point)to an integer.
*
* @param	FloatNum is the floating point number.
*
* @return	Integer number to a precision of 3 digits.
*
* @note
* This function is used in the printing of floating point data to a STDIO device
* using the xil_printf function. The xil_printf is a very small foot-print
* printf function and does not support the printing of floating point numbers.
*
*/
int float2Int(float f)
{
	return XAdcFractionToInt(f);
}
int XAdcFractionToInt(float FloatNum)
{
	float Temp;

	Temp = FloatNum;
	if (FloatNum < 0)
	{
		Temp = -(FloatNum);
	}

	return( ((int)((Temp -(float)((int)Temp)) * (1000.0f))));
}

/*******************************************************/
/* Combining data for bram */
int CombineForBram(int battery_thresh, int pwm)
{
	int result_pwm = 0b011;
	int result_bat = 0b001;
	int result;
	result_pwm = result_pwm & pwm;

	result_bat = result_bat & battery_thresh;
	result_bat = result_bat << 2;

	result = result_pwm | result_bat;

	return result;
}

\end{lstlisting}

\end{document}